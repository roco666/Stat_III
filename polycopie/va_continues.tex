\chapter{Distributions continues}

Les donn\'ees peuvent provenir d'\'ev\'enements ponctuels comme le nombre de personnes entrant dans une banque en 1 heure, ou le nombre d'appels t\'el\'ephonique \`a faire afin de trouver 10 acheteurs d'un nouveau produit. Mais souvent, les donn\'ees sont continues, comme la quantit\'e de panure sur les mets \`a base de poisson, la longueur de coupes d'aciers, le temps de recharge d'une batterie de t\'el\'ephone, $\ldots$ .

Afin de mod\'eliser ces ph\'enom\`enes de type continu, nous utilisons des distributions continues, comme
\begin{itemize}
	\item la loi uniforme
	\item la loi normale
	\item la loi exponentielle
	\item ....
\end{itemize}

Des \emph{tables statistiques} sont disponibles pour chacune de ces lois, ainsi que des fonctions associ\'ees dans les logiciels de bureautique (MS Excel, Calc d'Openoffice, Gnumeric, ...) ou sp\'ecialis\'es (R).

\subsubsection{Variable al\'eatoire continue}
Une variable al\'eatoire continue $X$ prend ses valeurs dans un intervalle qui est un sous-ensemble de l'ensemble des r\'eels $\IR$ :
$$X \in [u,v]$$
avec $u,v\in\IR, u < v$

Le nombre de valeurs possibles de $X$ \'etant infini, chacune de ces valeurs a une probabilit\'e nulle. En revanche, il est possible de calculer la probabilit\'e associ\'ee \`a n'importe quel sous-intervalle $[a, b]$ de $[v,w]$ :
$$\begin{array}{ll}
	P(X = x) = 0 & \forall x \in [v,w]\\
  P(X \in [a, b]) \geq 0 & \forall a < b, [a, b] \subseteq [v,w]$$
\end{array}$$

\subsubsection{Fonction de densit\'e}
Une distribution continue est d\'efinie soit par sa fonction de densit\'e, soit par sa fonction de r\'epartition.

Une distribution continue peut \'etre repr\'esent\'ee par un histogramme. Si l'on dispose d'un grand nombre d'observations, il est possible de d\'efinir des classes de tr\`es petites amplitudes. A la limite, les classes deviennent d'amplitude nulle et l'histogramme devient une courbe. Cette courbe est la fonction de densit\'e de la variable al\'eatoire continue X, not\'ee $f(x)$.

La fonction de densit\'e n'est pas une distribution de probabilit\'e. En effet, pour tout $x$ appartenant \`a l'intervalle $[v,w]$, $P(X = x) = 0$, mais $f(x) \geq 0$.

En revanche, l'aire sous la courbe d'une fonction de densit\'e doit valoir 1, qui est la somme de toutes les probabilit\'es. Ainsi, la relation suivante est v\'erifi\'ee pour toute fonction de densit\'e
$$P(X \in ] -\infty,\infty[) = 1$$

\subsubsection{Fonction de r\'epartition}
La probabilit\'e d'\'etre dans un intervalle de longueur $dx$ tr\`es petit est donn\'ee par $f(x) dx$ :
$$P(X \in [x, x + dx]) = f(x) dx$$

La probabilit\'e de se trouver dans un intervalle $[a, b]$ est d\'efinie comme l'aire sous la fonction de densit\'e.
Math\'ematiquement, cela revient \`a calculer l'int\'egrale de la fonction $f(x)$ entre $a$ et $b$ :
$$P(X \in [a, b]) = \int^b_a f(x) dx$$

\begin{defi}
La fonction de r\'epartition, not\'ee $F(a)$, exprime pour une variable al\'eatoire $X$ la probabilit\'e d'\'etre inf\'erieure ou \'egale \`a $x$. Cela correspond \`a la surface sous la fonction de densit\'e \`a gauche de $x$ :

$$F(a) = P(X \leq a) = \int^a_{-\infty} f(x) dx$$
\end{defi}

\begin{pro}
La fonction de r\'epartition est telle que
\begin{itemize}
	\item $F(-\infty) = 0$, 
	\item $F(\infty) = 1$
	\item $P(X <a) = P(X \leq a) = F(a)$
  \item $P(X \in [a, b]) = F(b) - F(a)$
  \item Par compl\'ementarit\'e, $P(X>a) = 1- P(X\leq a) = 1-F(a)$
\end{itemize}
\end{pro}

\subsubsection{Esp\'erance et variance}

\begin{defi}
L'\emph{esp\'erance math\'ematique} d'une variable al\'eatoire continue $X$ le nombre
$$\E(X)=\int^{\infty}_{-\infty} x f(x) dx$$
\end{defi}

\begin{defi}
La \emph{variance} d'une variable al\'eatoire continue $X$ le nombre
$$\var(X)=\E\left[ X-\E(X)\right]^2 = \E(X^2)-\left[ \E(X)\right]^2$$
\end{defi}
