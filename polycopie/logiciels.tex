\chapter{Instructions pour logiciels}

Voici quelques instructions pour quelques logiciels disponibles sur les ordinateurs de la HEG Gen\`eve.

\section{Loi normale}
\begin{tabular}{lc|l@{ = }l}
	Logiciel & Version \\
	\hline
	Calc d'OpenOffice & 2.4 & $z_{\alpha}$ & {\sc loi.normale.standard.inverse}()\\
	 &  & $p$ & {\sc loi.normale}()\\
	MS Excel & 2010 & $z_{\alpha}$ & {\sc loi.normale.inverse.n}()\\
	 &  & $p$ & {\sc loi.normale.n}()\\
	R & 2.13 & $z_{\alpha}$ & qnorm()\\
	  &       & $p$          & pnorm()\\
\end{tabular}

\begin{ex}
Calculer les valeurs critiques $z_{\alpha/2}$ pour un degr\'e de confiance $1-\alpha=0.9$ dans un test bilat\'eral, i.e. tels que $P(z_{\alpha/2})=0.9$

\begin{tabular}{|rl|}
\hline
	R & qnorm(0.1/2) ; qnorm(1-0.1/2)\\
	Ms Excel & {\sc =loi.normale.standard.inverse.n(0.1/2)}\\
	Calc & = {\sc loi.normale.standard.inverse(0.1/2)}\\
\hline
\end{tabular}
\end{ex}

\section{$t$-distribution}
\begin{tabular}{lc|l@{ = }l}
	Logiciel & Version \\
	\hline
	Calc d'OpenOffice & 2.4 & $t_{\alpha}$ & {\sc loi.student.inverse}()\\
	 &  & $p$ & {\sc loi.student}()\\
	MS Excel & 2010 & $t_{\alpha}$ & {\sc loi.student.inverse.n}()\\
	 &  & $p$ & {\sc loi.student.n}()\\
	R & 2.13 & $t_{\alpha}$ & qt()\\
	  &       & $p$          & pt()\\
\end{tabular}

\begin{ex}
Calculer les valeurs critiques pour un degr\'e de confiance $1-\alpha=0.95$ dans un test unilat\'eral avec 10 degr\'es de libert\'e.

\begin{tabular}{|rl|}
\hline
	R & qt(0.95, 10)\\
	Ms Excel & {\sc =loi.student.inverse.n(0.95;10)}\\
	Calc & {\sc =loi.student.inverse(2*0.05;10)}\\
\hline
\end{tabular}
\end{ex}


\section{$\chi^2_n$ distribution}
\begin{tabular}{lc|l@{ = }l}
	Logiciel & Version\\
	\hline
	Calc d'OpenOffice & 2.4 & $\ki^2_{1-\alpha}$ & {\sc khideux.inverse}()\\
	 &  & $p$ & 1-{\sc loi.khideux}()\\
	MS Excel & 2010 & $\ki^2_{1-\alpha}$ & {\sc loi.khideux.inverse}()\\
	 &  & $p$ & 1-{\sc loi.khideux.n}()\\
	R & 2.13 & $\ki^2_{\alpha}$ & {\sc qchisq}()\\
	  &       & $p$          & pchisq()\\
\end{tabular}

\begin{ex}
Calculer la valeur critique pour un degr\'e de confiance $1-\alpha=0.9$ dans un test unilat\'eral avec 19 degr\'es de libert\'e.

\begin{tabular}{|rl|}
\hline
	R & qchisq(0.9,19)\\
	Ms Excel & {\sc =loi.khideux.inverse.(0.9;19)}\\
	Calc & {\sc =khideux.inverse(0.1;19)}\\
\hline
\end{tabular}
\end{ex}

\section{Intervalle de confiance}
La fonction {\sc intervalle.confiance} d'Excel et Calc d'OpenOffice ne calculent pas un intervalle de confiance, mais la marge d'erreur dans le cas suivant: la population suit une loi normale, et son \'ecart type est connu.

\begin{tabular}{lc|l}
	Logiciel & Version \\
	\hline
	Calc d'OpenOffice & 2.4 & $\mbox{Marge}_z$ = {\sc intervalle.confiance}()\\
	MS Excel & 2010 & $\mbox{Marge}_z$ = {\sc intervalle.confiance}()\\
	R & 2.13 & $IC_t$ = t.test()\$conf.int\\
\\
\end{tabular}

\begin{ex}
Calculer la marge de l'IC pour une moyenne, dont l'\'ecart type connu de la population est 0.04, le degr\'e de confiance vaut 0.95, la taille de l'\'echantillon est 4 et la population suit une loi normale.\\[5mm]
\begin{tabular}{|rl|}
\hline
	Ms Excel & {\sc =intervalle.confiance(0.05;0.04;4)}\\
	Calc & {\sc =intervalle.confiance(0.05;0.04;4)}\\
\hline
\end{tabular}

\vspace{5mm}
Reprenons l'exemple en \ref{exICMoyenneStudent}\\[5mm]
\begin{tabular}{|rl|}
\hline
	R:  x <- & c(7.1,13.6,1.4,3.6,1.9,11.6,1.7,16.9,2.6,7.7,\\
	         & 12.4,11,3.7,14.6,8.8,8.5,6.1,3.3,6.1,6.9,0.4,11,0.8,6.4,9.1)
\\
 & t.test(x)\$conf.int\\
\hline
\end{tabular}
\end{ex}

\section{Test d'une moyenne}
\begin{tabular}{lc|l}
	Logiciel & Version \\
	\hline
	Ms Excel & 2010 & {\sc =test.student}()\\
	R & 2.13 & t.test\\
	 & & library(TeachingDemos); z.test()\\
\\
\end{tabular}

\begin{ex}
Test unilat\'eral de la moyenne:  $H_0: \mu\leq 5.2$ lorsque la variance est inconnue, pour un \'echantillon de taille 15, et un seuil de signification de 0.05.

\begin{tabular}{|rl|}
\hline
	Ms Excel & {\sc =test.student(x,5.2;1;3)}\\
	R:  & x <- c(4,4.1,6,5.5,5.8,4.2,3.1,6,4.9,3.9,4.8,5.6,5.7,5.3,5.8)\\
	    & t.test(x, mu = 5.2, alternative = "greater", correct=FALSE)\\
\hline
\end{tabular}
\end{ex}

\begin{ex}
Test bilat\'eral de la moyenne:  $H_0: \mu= 99$ lorsque l'\'ecart type est connu ($\sigma = 5$) , avec un seuil de signification de 0.05.

\begin{tabular}{|rl|}
\hline
	R:  & x <- rnorm(25, 100, 5)\\
	    & library(TeachingDemos); z.test(x, mu = 99, stdev= 5)\\
\hline
\end{tabular}
\end{ex}

\section{Test d'une proportion}
\begin{tabular}{lc|l}
	Logiciel & Version \\
	\hline
	R & 2.9.1 & prop.test\\
\\
\end{tabular}

\begin{ex}
Reprenons l'exemple \ref{exTestProportion}, dans lequel 9 contrats sur 600 \'etaient incomplets. La banque exige qu'il n'y ait pas plus d'1\% de contrats incomplets, avec un niveau de signification fix\'e \`a 0.02. 

\begin{tabular}{|rl|}
\hline
	R:  & prop.test(9, 600, p=0.01, alternative="greater", conf.level = 0.98, correct=FALSE)\\
\hline
\end{tabular}
\end{ex}

\section{Test d'une variance}
 
\begin{ex}
Tester si la variance est est en dessous de 0.005.\\
Valeurs: -0.04,  0.11, -0.10,  0.05,  0.20, -0.05, -0.04,  0.13,  0.01,  0.05, -0.05,  0.04, -0.03,  0.11,  0.05, -0.03, -0.21,  0.19,  0.04, -0.14\\
H0: $\sigma^2\leq 0.005$\\
H1: $\sigma^2> 0.005$

\begin{verbatim}
y <- c( -0.04,  0.11, -0.10,  0.05,  0.20, -0.05, -0.04,  0.13,  0.01,  0.05,-0.05,  0.04, -0.03,  0.11,  0.05, -0.03, -0.21,  0.19,  0.04, -0.14)
sigmacarre = 0.005
p= pchisq( var(y)*(length(y)-1)/sigmacarre, length(y)-1,lower.tail=FALSE)
\end{verbatim}
\end{ex}

%\section{Test des rangs sign\'es de Wilcoxon}
%\begin{tabular}{lc|l@{ = }l}
%	Logiciel & Version \\
%	\hline
%	R & 2.9.1 &  & wilcox.test()
%\end{tabular}
%
%\begin{ex}
%Tester si la valeur de la m\'ediane est de 40.
%
%\begin{verbatim}
%x <- c(31, 48, 23, 56, 28, 29, 44)
%wilcox.test(x, alternative = "two.sided", mu=40, 
%            exact = TRUE, conf.level=0.9, conf.int = TRUE)
%\end{verbatim}
%\end{ex}

\section{Test d'ind\'ependance}
\begin{tabular}{lc|l@{ = }l}
	Logiciel & Version \\
	\hline
	R & 2.9.1 &  & chisq.test()
\end{tabular}

\begin{ex}
D\'eterminer si deux variables sont ind\'ependantes.

\begin{verbatim}
amendes <- matrix(c(240,160,80,40,32,18,11,9,5,4),ncol=5,byrow=TRUE)
rownames(amendes)<-c("Homme","Femme")
colnames(amendes)<-c("Vitesse","Parcage","Feux grillés","Service_AP", "Autre")
amendes <- as.table(amendes)
chisq.test(amendes)
\end{verbatim}
\end{ex}

\section{Test de corr\'elation lin\'eaire}
\begin{tabular}{lc|l@{ = }l}
	Logiciel & Version \\
	\hline
	R & 2.9.1 &  & cor.test()
\end{tabular}

\begin{ex}
D\'eterminer si deux variables sont positivement corr\'el\'ees (lin\'eairement).

\begin{verbatim}
taille <- c(90, 160, 250, 160, 200, 160, 200, 200, 160, 90)
proportion <- c(0.13, 0.16, 0.21, 0.18, 0.18, 0.19, 0.15, 0.17, 0.13, 0.11)
cor.test(taille, proportion, method = "pearson", alternative = "greater")
\end{verbatim}
\end{ex}

\section{R\'egression lin\'eaire}
\begin{tabular}{lc|l@{ = }l}
	Logiciel & Version \\
	\hline
	R & 2.9.1 &  & lm()
\end{tabular}

\begin{ex}
\begin{verbatim}
x <- c(3,5,2,8,2,6,7,1,4,2,9,6)
y <- c(487,445,272,641,187,440,346,238,312,269,655,563)
lm(y~x) # coefficients
summary(lm.D90 <- lm(y ~ x)) # avec tests
\end{verbatim}
\end{ex}
