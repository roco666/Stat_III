 \documentclass[a4paper,11pt]{article}

 \usepackage{a4e,tp_heg}
 \usepackage[french]{babel}
 \usepackage[T1]{fontenc}

 \parindent 0cm
 \begin{document}

 %------------------------------------------------------------------
 \begin{center}
     \fbox{\parbox{12cm}{\footnotesize\center\bfseries\vspace{-2ex}
			Dur�e de l'examen: 90 minutes.\\
	  		{* * *} \\
            Toute documentation autoris�e, � l'exception des ordinateurs.\\
            {* * *} \\
			Les r�ponses doivent �tre donn�es dans les cadres pr�vus � cet effet.\\
	  		{* * *} \\
            Il est indispensable de justifier les r�ponses.\\
        }   }
 \end{center}
 %-----------------------------------------------------------------



 \newpage

 {\bfseries EXERCICE 1}

 \vspace*{1ex} 
 
 Vous vous occupez du recrutement au sein d'une entreprise du b�timent. Chaque jour ouvrable, vous rencontrez un certain nombre de demandeurs
d'emploi. L'�chantillon suivant repr�sente le nombre de demandeurs que vous avez rencontr�s durant 9 jours s�lectionn�s au hasard durant less
jours ouvrables de l'ann�e 2007:

 \begin{center}
 4 \quad 2 \quad 1 \quad 6 \quad 6 \quad 4 \quad 8 \quad 4 \quad 3
 \end{center}


 \begin{enumerate}
    \item Valeur 8 points.\\ Construisez un intervalle de confiance � 95\% pour le nombre moyen de demandeurs d'emploi rencontr�s
quotidiennement. Commentez le r�sultat.

 \vspace*{1ex}

       \definboxwidth
       \framebox[\linewidth][l] {\parbox{\inboxwidth}{
       \vspace{10cm}
       }}
 \vspace*{1ex}
 
    \item Valeur 1 point.\\ Selon vous, est-il courant de rencontrer plus de 7 demandeurs d'emploi un m�me jour? Justifiez votre r�ponse.

 \vspace*{1ex}

       \definboxwidth
       \framebox[\linewidth][l] {\parbox{\inboxwidth}{
       \vspace{4cm}
       }}
 \end{enumerate}

 \newpage

 {\bfseries EXERCICE 2}

 \vspace*{1ex}
 Valeur 8 points.\\
 Vous vous occupez du recrutement au sein d'une entreprise du b�timent. Chaque jour ouvrable, vous rencontrez un certain nombre de demandeurs
d'emploi. L'�chantillon suivant repr�sente le nombre de demandeurs que vous avez rencontr�s durant 10 jours s�lectionn�s au hasard durant les
jours ouvrables de l'ann�e 2005:

 \begin{center}
 6 \quad 7 \quad 1 \quad 7 \quad 6 \quad 3 \quad 8 \quad 9 \quad 3 \quad 7
 \end{center}


 Nous aimerions tester l'hypoth�se selon laquelle la variance du nombre de demandeurs d'emploi rencontr�s chaque
jour est �gale � 6. Effectuez un test bilat�ral avec un risque $\alpha$ de 5\% et commentez le r�sultat.

 \vspace*{1ex}

       \definboxwidth
       \framebox[\linewidth][l] {\parbox{\inboxwidth}{
       \vspace{14cm}
       }}
 


 \newpage

 {\bfseries EXERCICE 3}

 \vspace*{1ex}
 Valeur 5 points.\\
 Nous voulons � pr�sent savoir si le nombre moyen de demandeurs d'emploi rencontr�s quotidiennement a �volu� entre 2005 et 2007. Pour cela, nous
avons effectu� un test de Student pour la comparaison des moyennes. Le listing suivant a �t� obtenu � l'aide du logiciel R:
 \vspace*{1ex}


 \begin{verbatim}
 	Two Sample t-test
 
 data:  var3 by var4
 t = -1.3559, df = 17, p-value = 0.1929
 alternative hypothesis: true difference in means is not equal to 0
 95 percent confidence interval:
  -3.7772633  0.8217077
 sample estimates:
 mean in group 2007 mean in group 2005
           4.222222           5.700000
 \end{verbatim}

 \vspace*{1ex}

 Ecrivez les hypoth�ses nulle et alternative, concluez en fonction du listing et commentez l'ensemble des r�sultats fournis. Le
risque de premi�re esp�ce, $\alpha$, est toujours fix� � 5\%.

 \vspace*{1ex}

       \definboxwidth
       \framebox[\linewidth][l] {\parbox{\inboxwidth}{
       \vspace{11cm}
       }}


 
 \end{document}
