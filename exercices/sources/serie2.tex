\documentclass[french,a4paper,12pt]{article}
%\usepackage[pdftex]{graphicx}
% Pour la c\'esure des mots
\usepackage[T1]{fontenc}
% Pour traduire par exemple Chapter -> Chapitre
\usepackage{babel}
\usepackage{hegexo}


\begin{document}
\cours{Statistique inf�rentielle}
\title{S�rie 2}
\author{Dr S. Varone}
\date{}

\maketitle

%%%%%%%%%%%%%%%%%%%%%%%%%%%%%%%%%%%%%%%%%%%%%%%%%%%%%%%
%%                                                   %% 
%%             COURS 2: Estimations                  %%
%%                                                   %% 
%%%%%%%%%%%%%%%%%%%%%%%%%%%%%%%%%%%%%%%%%%%%%%%%%%%%%%%

\section{Nombre de spectateurs} % Estimations de la moyenne et de la variance 
% Berchtold ex 3.1
Soit $Y$ le nombre de personnes qui assistent � la repr�sentation d'une pi�ce de th��tre.
 \begin{enumerate}
 \item Pour 5 jours de repr�sentation pris au hasard on a observ�:
	 $$
	 \begin{array}{l|ccccc}
	 i & 1 & 2 & 3 & 4 & 5\\
	 \hline
	 y_i & 300 & 280 & 290 & 310 & 295
	 \end{array}
	 $$
	 Estimer l'esp�rance $E(Y) = \mu $ par la m�diane de l'�chantillon 	 $\tilde{y}$, puis par la moyenne de l'�chantillon $\bar{y}$.
 \item Pour 4 nouveaux jours choisis au hasard on a observ� :
	 $$
	 \begin{array}{l|cccc}
	 i & 6 & 7 & 8 & 9\\
	 \hline
	 y_i & 305 & 318 & 290 & 280
	 \end{array}
	 $$
	 Recalculer les deux estimations de $\mu $ en consid�rant l'�chantillon form� par l'ensemble des 9 observations. Commenter.
 \item Pour les cinq premi�res observations on trouve:
	 $$
	 \sum\limits^5_{i=1} y^2_i = 435'625
	 $$
	 et pour l'ensemble des 9 observations:
	 $$
	 \sum\limits^9_{i=1} y^2_i = 792'274
	 $$
	 Donner une estimation non-biais�e de la variance de $Y$, ainsi qu'une estimation
	non-biais�e de l'�cart type de $\bar{y}$ dans chacun des deux cas consid�r�s.
	Commenter.
 \item Supposons maintenant qu'� la suite d'une erreur le nombre d'entr�es du 9�me jour ait �t� mal relev�. De ce fait nous avons:
	 $$
	 \begin{array}{l|ccccccccc}
	 i & 1 & 2 & 3 & 4 & 5 & 6 & 7 & 8 & 9\\
	 \hline
	 y_i & 300 & 280 & 290 & 310 & 295 & 305 & 318 & 290 & 28
	 \end{array}
	 $$
	 Recalculer les deux estimations de $\mu $. Lequel des deux estimateurs est-il le plus	pertinent?
 \end{enumerate}

\end{document}