\documentclass[french,a4paper,12pt]{article}
\usepackage[dvips]{graphicx}
% Pour la c\'esure des mots
\usepackage[T1]{fontenc}
% Pour traduire par exemple Chapter -> Chapitre
\usepackage{babel}
\usepackage{hegexo}


\begin{document}
\cours{Statistique inf�rentielle}
\title{S�rie 6}
\author{Dr S. Varone}
\date{}

\maketitle

%%%%%%%%%%%%%%%%%%%%%%%%%%%%%%%%%%%%%%%%%%%%%%%%%%%%%%%
%%                                                   %% 
%%     COURS 6: IC                                   %%
%%                                                   %% 
%%%%%%%%%%%%%%%%%%%%%%%%%%%%%%%%%%%%%%%%%%%%%%%%%%%%%%%

\section{Importation} % IC proportion
% adapt� de \cite{groebner2005} ex 7.53
Les fabricants d'ordinateurs ach�tent des barrettes m�moire � Ta\"{i}wan au meilleur prix. Toutefois, il arrive que certaines importations ne sont pas de qualit� suffisante. Un lot vient d'arriver chez Dell d'un fournisseur de Ta\"{i}wan. Afin de contr�ler la qualit� du lot, 130 barrettes sont s�lectionn�es al�atoirement et test�es: 13 se sont r�v�l�es de qualit� insuffisante. 
\begin{enumerate}
	\item Donnez un intervalle de confiance � 90\% concernant la proportion de barrettes d�fectueuses dans le lot.
	% $0.1\pm 1.645\sqrt{\frac{0.1(1-0.1)}{130}} = 0.1\pm 0.433$ i.e. $[0.0567;0.1433]$
	\item Supposez que Dell souhaite r�duire la marge d'erreur pour l'estimation donn�e pr�c�demment. Quelles options a-t-il?
	% R�duire le degr� de confiance ou prendre un �chantillon de taille plus grande
\end{enumerate}


\section{Passagers par voiture*} % IC moyenne, variance
	% source Andre Berchtold B.3.4
Dans le but d'estimer le nombre moyen de passagers (conducteur
compris) par v�hicule automobile circulant sur l'autoroute
Gen�ve-Lausanne, un observateur a recueilli les donn�es suivantes:

 \begin{center}
 \begin{tabular}{lccccccr}
 \hline\hline
 Nombre de passagers & 1 & 2 & 3 & 4 & 5 & 6 & Total\\
 Effectif & 230 & 248 & 117 & 76 & 14 & 3 & 688\\
 \hline
 \end{tabular}
 \end{center}

\`A l'aide d'intervalles de confiance � 95\%
 \begin{enumerate}
	\item Estimer la moyenne $\mu$ de la population
	\item Estimer la variance $\sigma^2$ de la population
\end{enumerate}


\end{document}