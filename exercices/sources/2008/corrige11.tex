\documentclass[french,a4paper,12pt]{article}
\usepackage[dvips]{graphicx}
% Pour la c\'esure des mots
\usepackage[T1]{fontenc}
% Pour traduire par exemple Chapter -> Chapitre
\usepackage{babel, amsfonts}
\usepackage{hegexo}


\begin{document}
\cours{Statistique inf�rentielle}
\title{Corrig� 11}
\author{Dr. S. Varone}
\date{Semaine 48, 2008}

\maketitle

%%%%%%%%%%%%%%%%%%%%%%%%%%%%%%%%%%%%%%%%%%%%%%%%%%%%%%%
%%                                                   %% 
%%     COURS 11: Test de la m�diane                  %%
%%                                                   %% 
%%%%%%%%%%%%%%%%%%%%%%%%%%%%%%%%%%%%%%%%%%%%%%%%%%%%%%%

\section{Lampe de beamer} % Test de la m�diane
	% \cite{groebner2006} ex 16.5 p. 674
Un fabricant de beamer affirme que les lampes de ses produits ont une dur�e de vie m�diane de 4000 heures. Un magazine de consommateurs d�cide de v�rifier les dires de ce fabricant et teste la dur�e de vie de 12 beamers.

\begin{tabular}{*{6}{c}}
1973 & 4838 & 3805 & 4494 & 4738 & 5249\\
4459 & 4098 & 4722 & 5894 & 3322 & 4800
\end{tabular}

En utilisant un niveau de signification de 0.05, quelle conclusion le magazine de consommateurs peut-il avoir ?

Le test ne peut pas �tre fait sur une moyenne car la population ne suit pas une loi normale.

\begin{verbatim}
x <- c(1973, 4838, 3805, 4494, 4738, 5249, 4459, 4098, 4722, 5894, 3322, 4800)
hist(x)
wilcox.test(x, alternative = "two.sided", mu=4000, exact = TRUE, conf.level=0.95, conf.int = TRUE)
\end{verbatim}


	\begin{enumerate}
		\item Le param�tre est $\tilde{\mu}=4000$
		\item 
			\begin{enumerate}
				\item $\begin{array}{ll}
							  H_0 :& \tilde{\mu}\leq 4000 \\
							  H_1 :& \tilde{\mu} > 4000
							 \end{array}$
				\item $\begin{array}{ll}
							  H_0 :& \tilde{\mu}\geq 4000 \\
							  H_1 :& \tilde{\mu} < 4000
							 \end{array}$
				\item $\begin{array}{ll}
							  H_0 :& \tilde{\mu}= 4000 \\
							  H_1 :& \tilde{\mu}\not= 4000
							 \end{array}$
			\end{enumerate}
	  \item Le niveau de signification est $\alpha=0.05$
	  \item La r�gion critique est 
			  \begin{enumerate}
					\item $[61 ; \infty [$
					\item $[0 ; 17]$
					\item $\IN\backslash ]13; 65[$
				\end{enumerate}
	  \item Calcul de la statistique
	  
\begin{tabular}{*{6}{c}}
Donn�es & Diff�rences & Abs(Diff�rences) & Rang	& Rang + & 	Rang -\\
1973 & -2027 & 2027 & 12 &  & 12\\
4838 & 838 & 838 & 9 & 9 &\\
3805 & -195 & 195 & 2 & &  2\\
4494 & 494 & 494 & 4 & 4 & \\
4738 & 738 & 738 & 7 & 7 & \\
5249 & 1249 & 1249 & 10 & 10 & \\
4459 & 459 & 459 & 3 & 3 & \\
4098 & 98 & 98 & 1 & 1 & \\
4722 & 722 & 722 & 6 & 6 & \\
5894 & 1894 & 1894 & 11 & 11 & \\
3322 & -678 & 678 & 5 &  & 5\\
4800 & 800 & 800 & 8 & 8 & \\
 & & & & W=	59 &	
\end{tabular}
	  
	  \item 
		  \begin{enumerate}
				\item Comme $W=59<61=W_0$ l'hypoth�se $H_0$ n'est pas rejet�e
				\item Comme $W=59>17=W_0$ l'hypoth�se $H_0$ n'est pas rejet�e
				\item Comme $W=59\in [13,65]$ l'hypoth�se $H_0$ n'est pas rejet�e
			\end{enumerate}
	  \item La conclusion est qu'il n'y a pas suffisamment d'�vidence pour rejeter l'affirmation du fabricant
	\end{enumerate}


\end{document}