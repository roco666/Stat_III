\documentclass[french,a4paper,12pt]{article}
\usepackage[dvips]{graphicx}
% Pour la c\'esure des mots
\usepackage[T1]{fontenc}
% Pour traduire par exemple Chapter -> Chapitre
\usepackage{babel}
\usepackage{hegexo}


\begin{document}
\cours{Statistique inf�rentielle}
\title{Corrig� 3}
\author{Dr. S. Varone}
\date{Semaine 40, 2008}

\maketitle

%%%%%%%%%%%%%%%%%%%%%%%%%%%%%%%%%%%%%%%%%%%%%%%%%%%%%%%
%%                                                   %% 
%%             COURS 3: Estimations                  %%
%%                                                   %% 
%%%%%%%%%%%%%%%%%%%%%%%%%%%%%%%%%%%%%%%%%%%%%%%%%%%%%%%

\section{Nombre de spectateurs} % Estimations de la moyenne et de la variance 
% Berchtold ex 3.1
Soit $Y$ le nombre de personnes qui assistent � la repr�sentation d'une pi�ce de th��tre.
 \begin{enumerate}
 \item Pour 5 jours de repr�sentation pris au hasard on a observ�:
	 $$
	 \begin{array}{l|ccccc}
	 i & 1 & 2 & 3 & 4 & 5\\
	 \hline
	 y_i & 300 & 280 & 290 & 310 & 295
	 \end{array}
	 $$
	 Estimer l'esp�rance $E(Y) = \mu $ par la m�diane de l'�chantillon 	 $\tilde{Y}$, puis par la moyenne de l'�chantillon $\overline{Y}$.
	 $$\overline{Y}=\frac{1475}{5}=295\quad \tilde Y = 295$$
 \item Pour 4 nouveaux jours choisis au hasard on a observ� :
	 $$\begin{array}{l|cccc}
	 i & 6 & 7 & 8 & 9\\
	 \hline
	 y_i & 305 & 318 & 290 & 280
	 \end{array}$$
	 Recalculer les deux estimations de $\mu $ en consid�rant l'�chantillon form� par l'ensemble des 9 observations. Commenter.
	 $$\tilde Y = 295\quad\overline{Y}=\frac{2668}{9}\approx 296.4$$

 \item Pour les cinq premi�res observations on trouve:
	 $$\sum\limits^5_{i=1} y^2_i = 435'625$$
	 et pour l'ensemble des 9 observations:
	 $$\sum\limits^9_{i=1} y^2_i = 792'274$$
	 
	 Donner une estimation non-biais�e de la variance de $Y$, ainsi qu'une estimation
	non-biais�e de l'�cart type de $\overline{Y}$ dans chacun des deux cas consid�r�s.
	Commenter.
	 $$\hat \sigma ^2 = \frac{n}{n-1}S^2= \frac{n}{n-1}
	 \left( \frac{1}{n} \sum\limits^n_{i=1} y^2_i  -\bar y ^2\right)$$
	 Attention, il s'agit de $S^2$ qui est la variance d'une population.
	 
	 \begin{itemize}
	 \item[a) cinq observations]
		  $$ S^2 = \frac{435625}{5} -295^2=100$$
			$$ \hat \sigma ^2 = \frac{5}{5-1}100= 125$$
		  $$\hat \sigma  = \sqrt{125} \approx 11.18$$
		 	Pour la moyenne de l'�chantillon, nous avons
		  $$\hat \sigma_{\bar y_1} = \frac{ \hat \sigma}{\sqrt{n}}= \frac{11.18}{\sqrt{5}}=5$$
   \item[b) neuf observations]
      $$S^2 = \frac{792274}{9} -296.4^2\approx151.14$$
      $$\hat \sigma ^2 = \frac{9}{9-1}151.14\approx 170.03$$
      $$ \hat \sigma \approx \sqrt{170.03} \approx 13.04$$
 			Pour la moyenne de l'�chantillon, nous avons
			$$\hat \sigma_{\bar y_2} = \frac{ \hat \sigma}{\sqrt{n}}= \frac{13.04}{\sqrt{9}}\approx 4.35$$ 
	 \end{itemize}
 \item Supposons maintenant qu'� la suite d'une erreur le nombre d'entr�es du 9�me jour ait �t� mal relev�. De ce fait nous avons:
	 $$\begin{array}{l|ccccccccc}
	 i & 1 & 2 & 3 & 4 & 5 & 6 & 7 & 8 & 9\\
	 \hline
	 y_i & 300 & 280 & 290 & 310 & 295 & 305 & 318 & 290 & 28
	 \end{array}$$
	 Recalculer les deux estimations de $\mu $. Lequel des deux estimateurs est-il le plus	pertinent?
   $$\tilde Y = 295\quad \overline{Y}=\frac{2416}{9}\approx 268.4$$
   La m�diane est ici plus pertinente que la moyenne comme estimateur de $\mu$ car elle est plus robuste.	 
 \end{enumerate}

\end{document}