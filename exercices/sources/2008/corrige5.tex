\documentclass[french,a4paper,12pt]{article}
\usepackage[dvips]{graphicx}
% Pour la c\'esure des mots
\usepackage[T1]{fontenc}
% Pour traduire par exemple Chapter -> Chapitre
\usepackage{babel}
\usepackage{hegexo}
\usepackage{pictex}


\begin{document}
\cours{Statistique inf�rentielle}
\title{Corrig� 5}
\author{Dr. S. Varone}
\date{Semaine 42, 2008}

\maketitle

%%%%%%%%%%%%%%%%%%%%%%%%%%%%%%%%%%%%%%%%%%%%%%%%%%%%%%%
%%                                                   %% 
%%     COURS 5: IC                                   %%
%%                                                   %% 
%%%%%%%%%%%%%%%%%%%%%%%%%%%%%%%%%%%%%%%%%%%%%%%%%%%%%%%

\section{Livreurs de pizzas} % IC moyenne, variance inconnue, petit �chantillon
La pizzeria ''C Bon'' a un service de livraison de pizzas. Le propri�taire d�sire estimer le pourboire moyen re�u par les livreurs. Pour cela, il a pris un �chantillon de 12 livraisons et a not� les pourboires re�us.

\begin{tabular}{*{6}r}
2.25 & 2.5 & 2.25 & 2 & 2 & 1.5\\
0 & 2 & 1.5 & 2 & 3 & 1.5
\end{tabular}

\begin{enumerate}
	\item Quelle distribution devrait �tre utilis�e pour d�velopper un intervalle de confiance pour le pourboire moyen par livraison?
	\\Distribution de Student
	\item Quelle condition doit satisfaire la distribution pour pouvoir obtenir une valeur critique ?
	\\La distribution de la population doit �tre normale.
	\item V�rifier si la condition semble satisfaite � l'aide d'un graphique.
  \\quantiles = $q_1=1.5$, $q_2=2$, $q_3=2.25$
  
	 \begin{verbatim}
x <- c(2.25,2.5,2.25,2,2,1.5,0,2,1.5,2,3,1.5)
quantile(x,type=6)
setwd("P:/Mes Documents/Cours/Stat_III/fig"); require(graphics)
pictex(file = "s5ex1.tex")
layout(matrix(1:2,1,2)); boxplot(x);hist(x)
dev.off()
	 \end{verbatim}

\begin{figure}[h]
  \centerline{\hbox{\beginpicture
\setcoordinatesystem units <1pt,1pt>
\setplotarea x from 0 to 361.35, y from 0 to 289.08
\setlinear
\font\picfont cmss10\picfont
\font\picfont cmss10 at 10pt\picfont
\font\picfont cmss10 at 10pt\picfont
\setdashpattern <15pt, 15pt, 15pt, 15pt, 15pt, 15pt, 15pt, 15pt>
\plot 75.10 162.54 105.57 162.54 /
\plot 105.57 162.54 105.57 209.46 /
\plot 105.57 209.46 75.10 209.46 /
\plot 75.10 209.46 75.10 162.54 /
\setsolid
\plot 75.10 193.82 105.57 193.82 /
\setdashpattern <4pt, 4pt>
\plot 90.34 256.37 90.34 209.46 /
\setsolid
\plot 82.72 162.54 97.96 162.54 /
\setsolid
\plot 82.72 256.37 97.96 256.37 /
\setsolid
\plot 75.10 162.54 105.57 162.54 /
\plot 105.57 162.54 105.57 209.46 /
\plot 105.57 209.46 75.10 209.46 /
\plot 75.10 209.46 75.10 162.54 /
\circulararc 360 degrees from 90.34 70.96 center at 90.34 68.71
\setsolid
\plot 49.20 68.71 49.20 256.37 /
\setsolid
\plot 49.20 68.71 43.20 68.71 /
\setsolid
\plot 49.20 99.98 43.20 99.98 /
\setsolid
\plot 49.20 131.26 43.20 131.26 /
\setsolid
\plot 49.20 162.54 43.20 162.54 /
\setsolid
\plot 49.20 193.82 43.20 193.82 /
\setsolid
\plot 49.20 225.10 43.20 225.10 /
\setsolid
\plot 49.20 256.37 43.20 256.37 /
\put {\rotatebox{90}{0.0} } [rB] <0.00pt,0.00pt> at 37.20 62.32
\put {\rotatebox{90}{0.5} } [rB] <0.00pt,0.00pt> at 37.20 93.60
\put {\rotatebox{90}{1.0} } [rB] <0.00pt,0.00pt> at 37.20 124.87
\put {\rotatebox{90}{1.5} } [rB] <0.00pt,0.00pt> at 37.20 156.15
\put {\rotatebox{90}{2.0} } [rB] <0.00pt,0.00pt> at 37.20 187.43
\put {\rotatebox{90}{2.5} } [rB] <0.00pt,0.00pt> at 37.20 218.71
\put {\rotatebox{90}{3.0} } [rB] <0.00pt,0.00pt> at 37.20 249.98
\setsolid
\plot 49.20 61.20 131.47 61.20 /
\plot 131.47 61.20 131.47 263.88 /
\plot 131.47 263.88 49.20 263.88 /
\plot 49.20 263.88 49.20 61.20 /
\font\picfont cmssbx10 at 12pt\picfont
\put {Histogram of x}  [lB] <0.00pt,0.00pt> at 231.19 269.28
\font\picfont cmss10 at 10pt\picfont
\put {x}  [lB] <0.00pt,0.00pt> at 268.71 13.20
\put {\rotatebox{90}{Frequency} } [rB] <0.00pt,0.00pt> at 193.87 141.26
\setsolid
\plot 232.92 61.20 309.10 61.20 /
\setsolid
\plot 232.92 61.20 232.92 55.20 /
\setsolid
\plot 245.62 61.20 245.62 55.20 /
\setsolid
\plot 258.32 61.20 258.32 55.20 /
\setsolid
\plot 271.01 61.20 271.01 55.20 /
\setsolid
\plot 283.71 61.20 283.71 55.20 /
\setsolid
\plot 296.41 61.20 296.41 55.20 /
\setsolid
\plot 309.10 61.20 309.10 55.20 /
\put {0.0}  [lB] <0.00pt,0.00pt> at 226.53 37.20
\put {1.0}  [lB] <0.00pt,0.00pt> at 251.93 37.20
\put {2.0}  [lB] <0.00pt,0.00pt> at 277.32 37.20
\put {3.0}  [lB] <0.00pt,0.00pt> at 302.71 37.20
\setsolid
\plot 229.87 68.71 229.87 256.37 /
\setsolid
\plot 229.87 68.71 223.87 68.71 /
\setsolid
\plot 229.87 115.62 223.87 115.62 /
\setsolid
\plot 229.87 162.54 223.87 162.54 /
\setsolid
\plot 229.87 209.46 223.87 209.46 /
\setsolid
\plot 229.87 256.37 223.87 256.37 /
\put {\rotatebox{90}{0} } [rB] <0.00pt,0.00pt> at 217.87 66.21
\put {\rotatebox{90}{1} } [rB] <0.00pt,0.00pt> at 217.87 113.12
\put {\rotatebox{90}{2} } [rB] <0.00pt,0.00pt> at 217.87 160.04
\put {\rotatebox{90}{3} } [rB] <0.00pt,0.00pt> at 217.87 206.96
\put {\rotatebox{90}{4} } [rB] <0.00pt,0.00pt> at 217.87 253.87
\setsolid
\plot 232.92 68.71 232.92 115.62 /
\plot 232.92 115.62 245.62 115.62 /
\plot 245.62 115.62 245.62 68.71 /
\plot 245.62 68.71 232.92 68.71 /
\setsolid
\plot 245.62 68.71 245.62 68.71 /
\plot 245.62 68.71 258.32 68.71 /
\plot 258.32 68.71 258.32 68.71 /
\plot 258.32 68.71 245.62 68.71 /
\setsolid
\plot 258.32 68.71 258.32 209.46 /
\plot 258.32 209.46 271.01 209.46 /
\plot 271.01 209.46 271.01 68.71 /
\plot 271.01 68.71 258.32 68.71 /
\setsolid
\plot 271.01 68.71 271.01 256.37 /
\plot 271.01 256.37 283.71 256.37 /
\plot 283.71 256.37 283.71 68.71 /
\plot 283.71 68.71 271.01 68.71 /
\setsolid
\plot 283.71 68.71 283.71 209.46 /
\plot 283.71 209.46 296.41 209.46 /
\plot 296.41 209.46 296.41 68.71 /
\plot 296.41 68.71 283.71 68.71 /
\setsolid
\plot 296.41 68.71 296.41 115.62 /
\plot 296.41 115.62 309.10 115.62 /
\plot 309.10 115.62 309.10 68.71 /
\plot 309.10 68.71 296.41 68.71 /
\endpicture
}
}
  \caption{}
\end{figure}
	\item Construire un intervalle de confiance � 90\% pour le pourboire moyen par livraison.
	\\$1.875 \pm 1.7959(.735/\sqrt{12}) =  1.875 \pm .3810 =   [1.494 ; 2.256]$
	\\ Attention: ne pas prendre $t^{11}_{\alpha}$ mais bien $t^{11}_{\alpha/2}$ 
	\\ $\sum{x_i-\bar{x}}=5.9375$, $s^2=0.5397727$, $s=0.7346923$, $\frac{s}{\sqrt{n}}=0.2120874$

	 \begin{verbatim}
t.test(x, conf.level=0.9)$conf.int
	 \end{verbatim}

\end{enumerate}

\section{Interpr�tations}
Une compagnie d'assurances a command� une �tude pour estimer le nombre moyen de km parcourus par ann�e par leurs clients ayant souscrits une police d'assurance voiture. L'�tude, bas�e sur un �chantillon al�atoire de 200 clients, a conclu que le nombre moyen de km parcourus se situait entre 8000 et 13000 km par ann�e. Cette estimation fut faite avec niveau de confiance de 95\%.
\begin{enumerate}
	\item Un manager dit � son coll�gue que le 95\% de leurs clients parcourent entre 8000 et 13000 km par ann�e. Cette affirmation est-elle correcte? Justifiez.
	\\Non, 0.95 est une probabilit�, pas une proportion.
	\item Un autre manager r�sume en s�ance l'�tude command�e en affirmant ''La compagnie peut �tre s�r � 95\%  que le nombre moyen de km parcourus en une ann�e par les 200 clients de l'�chantillon, se trouve entre 8000 et 13000km''. Cette affirmation est-elle correcte? Justifiez.
  \\Non, l'IC porte sur la moyenne de la population, pas de l'�chantillon
	\item Un troisi�me manager affirme qu'il y a 95\% de chance que le v�ritable nombre moyen de km parcourus par tous leurs clients se situe entre 8000 et 13000 km par ann�e. Cette affirmation est-elle correcte? Justifiez. 
	\\Non, le nombre moyen de km est une valeur et n'a donc pas de probabilit�.
\end{enumerate}

La seule interpr�tation correcte est de dire que parmi tous les IC possibles construits � partir d'�chantillons al�atoires de taille 200, le 95\% d'entre eux recouvrent la vraie valeur moyenne.


\end{document}