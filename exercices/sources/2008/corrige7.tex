\documentclass[french,a4paper,12pt]{article}
\usepackage[dvips]{graphicx}
% Pour la c\'esure des mots
\usepackage[T1]{fontenc}
% Pour traduire par exemple Chapter -> Chapitre
\usepackage{babel}
\usepackage{hegexo}


\begin{document}
\cours{Statistique inf�rentielle}
\title{Corrig� 7}
\author{Dr. S. Varone}
\date{Semaine 44, 2008}

\maketitle

%%%%%%%%%%%%%%%%%%%%%%%%%%%%%%%%%%%%%%%%%%%%%%%%%%%%%%%
%%                                                   %% 
%%     COURS 7: Test d'hypoth�se                     %%
%%                                                   %% 
%%%%%%%%%%%%%%%%%%%%%%%%%%%%%%%%%%%%%%%%%%%%%%%%%%%%%%%

\section{Formulation d'hypoth�ses I}
Un responsable de production d'Emmental doit v�rifier que le processus de fabrication du fromage Emmental utilise en moyenne 12 litres de lait pour 1kg de fromage. Si le nombre moyen de litres de lait est trop �lev�, les pertes sont au pr�judice de l'entreprise. Si le nombre moyen de litre de lait est trop faible, le fromage n'aura pas une qualit� suffisante. 
\begin{enumerate}
	\item Sur quelle valeur doit-il effectuer un test ?
\\Sur le nombre moyen de litres de lait par kg de fromage
	\item Quelle situation est suppos�e �tre vraie, tant qu'il n'y a pas suffisamment d'�vidence du contraire ?
\\$\mu=12$ litres de lait 
	\item Formuler les hypoth�ses nulle et alternative.
\\$H_0$ : $\mu=12$\\
\\$H_1$ : $\mu\not= 12$
\end{enumerate}

\section{Formulation d'hypoth�ses II}
Une zone agricole est utilis�e pour exp�rimenter un nouvel engrais, qui devrait permettre de diminuer la variation de rendement des cultures. Sur les 8 ann�es pr�c�dentes, il a �t� observ� des variations de $\pm 10\%$ des rendements. 
\begin{enumerate}
	\item Sur quelle valeur devra-t-on effectuer un test ?
\\Sur la variance des rendements
	\item Quelle situation est suppos�e �tre vraie, tant qu'il n'y a pas suffisamment d'�vidence du contraire ?
\\$\sigma^2=100$ 
	\item Formuler les hypoth�ses nulle et alternative.
\\$H_0$ : $\sigma^2\geq 100$\\
\\$H_1$ : $\sigma^2 < 100$
\end{enumerate}


\section{Erreur de premi�re et deuxi�me esp�ce} % cf exo Andre Berchtold 4.6
Supposons que la facture de chauffage mensuelle d'un petit appartement suive une loi normale de moyenne 100 Sfr et d'�cart type 20. L'isolation ayant �t� refaite, la charge mensuelle de chauffage est esp�r�e suivre une loi $\norm (80, 400)$.

Il s'agit de tester si la variable al�atoire suit encore une loi normale $\norm(100,400)$. La situation esp�r�e est la suivante $X\sim \norm(80,400)$\\[5mm]
$H_0$ : $\mu_0\geq 100$\\
$H_1$ : $\mu_0<100$

En choisissant une erreur de premi�re esp�ce $\alpha=0.05$, 
\begin{enumerate}
	\item quel est la valeur critique $r$?\\
		\begin{eqnarray*}
		x_c &=& \mu_0+ z_\alpha\cdot \sigma_0
		\\&=& \mu_0+ z_{0.05}\cdot \sigma_0
		\\&=&100+(-1.645)\cdot 20
		\\&=&67.1
		\end{eqnarray*}
	\item quel est la probabilit� de commettre une erreur de deuxi�me esp�ce ?\\
		\begin{eqnarray*}
		\beta &=& P(x_1> x_c \quad | \quad x_1 \;\sim\; N(\mu_1=80,\sigma^2_1 =400))
		\\&&\\&=& P(z_1> \frac{x_c-80}{20} \quad | \quad z_1 \;\sim\; N(0,1))
		\\&&\\&=& P(z_1> \frac{67.1-80}{20})
		\\&&\\&=& P(z_1> -0.645)
		\\&&\\&\approx& 0.74
		\end{eqnarray*}
\end{enumerate}

En choisissant une erreur de premi�re esp�ce $\alpha=0.01$, 
\begin{enumerate}
	\item quel est la valeur critique $r$?\\
		\begin{eqnarray*}
		x_c &=& \mu_0+ z_\alpha\cdot \sigma_0
		\\&=& \mu_0+ z_{0.01}\cdot \sigma_0
		\\&=&100+(-2.33)\cdot 20
		\\&=&53.4
		\end{eqnarray*}
	\item quel est la probabilit� de commettre une erreur de deuxi�me esp�ce ?\\
		\begin{eqnarray*}
		\beta &=& P(x_1> x_c \quad | \quad x_1 \;\sim\; N(\mu_1=80,\sigma^2_1 =400))
		\\&&\\&=& P(z_1> \frac{x_c-80}{20} \quad | \quad z_1 \;\sim\; N(0,1))
		\\&&\\&=& P(z_1> \frac{53.4-80}{20})
		\\&&\\&=& P(z_1> -1.33)
		\\&&\\&=& 0.9082
		\end{eqnarray*}
\end{enumerate}

\end{document}