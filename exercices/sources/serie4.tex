\documentclass[french,a4paper,12pt]{article}
\usepackage[dvips]{graphicx}
% Pour la c\'esure des mots
\usepackage[T1]{fontenc}
% Pour traduire par exemple Chapter -> Chapitre
\usepackage{babel}
\usepackage{hegexo}


\begin{document}
\cours{Statistique inf�rentielle}
\title{S�rie 4}
\author{Dr S. Varone}
\date{}

\maketitle

%%%%%%%%%%%%%%%%%%%%%%%%%%%%%%%%%%%%%%%%%%%%%%%%%%%%%%%
%%                                                   %% 
%%     COURS 4: IC                                   %%
%%                                                   %% 
%%%%%%%%%%%%%%%%%%%%%%%%%%%%%%%%%%%%%%%%%%%%%%%%%%%%%%%

\section{Livreurs de pizzas} % IC moyenne, variance inconnue, petit �chantillon
La pizzeria ''C Bon'' a un service de livraison de pizzas. Le propri�taire d�sire estimer le pourboire moyen re�u par les livreurs. Pour cela, il a pris un �chantillon de 12 livraisons et a not� les pourboires re�us.

\begin{tabular}{*{6}r}
2.25 & 2.5 & 2.25 & 2 & 2 & 1.5\\
0 & 2 & 1.5 & 2 & 3 & 1.5
\end{tabular}

\begin{enumerate}
	\item Quelle distribution devrait �tre utilis�e pour d�velopper un intervalle de confiance pour le pourboire moyen par livraison?
	% Distribution de Student
	\item Quelle condition doit satisfaire la distribution pour pouvoir obtenir une valeur critique ?
	% La distribution de la population doit �tre normale.
	\item Dessiner un boxplot afin de v�rifier si la condition semble satisfaite.
	\item Construire un intervalle de confiance � 90\% pour le pourboire moyen par livraison.
	% $1.875 \pm 1.7959(.735/\sqrt{12}) =  1.875 \pm .3810 =  \[ 1.494 ----------- 2.2688\]$
\end{enumerate}

\section{Interpr�tations}
Une compagnie d'assurances a command� une �tude pour estimer le nombre moyen de km parcourus par ann�e par leurs clients ayant souscrit une police d'assurance voiture. L'�tude, bas�e sur un �chantillon al�atoire de 200 clients, a conclu que le nombre moyen de km parcourus se situait entre 8000 et 13000 km par ann�e. Cette estimation fut faite avec un niveau de confiance de 95\%.
\begin{enumerate}
	\item Un manager dit � son coll�gue que le 95\% de leurs clients parcourt entre 8000 et 13000 km par ann�e. Cette affirmation est-elle correcte? Justifiez.
	% Non, le nombre moyen de km est une valeur et n'a donc pas de probabilit�.
	\item Un autre manager r�sume en s�ance l'�tude command�e en affirmant ''La compagnie peut �tre s�re � 95\%  que le nombre moyen de km parcourus en une ann�e par les 200 clients de l'�chantillon, se trouve entre 8000 et 13000km''. Cette affirmation est-elle correcte? Justifiez.
	% Non, l'IC porte sur la moyenne de la population, pas de l'�chantillon
	\item Un troisi�me manager affirme qu'il y a 95\% de chance que le v�ritable nombre moyen de km parcourus par tous leurs clients se situe entre 8000 et 13000 km par ann�e. Cette affirmation est-elle correcte? Justifiez. 
	% Non, le nombre moyen de km est une valeur et n'a donc pas de probabilit�.
\end{enumerate}
% La seule interpr�tation correcte est de dire que sur parmi tous les IC possibles construits � partir d'�chantillons al�atoires de taille 200, le 95\% d'entre eux recouvrent la vraie valeur moyenne.

\end{document}