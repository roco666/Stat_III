\documentclass[french,a4paper,12pt]{article}
\usepackage[dvips]{graphicx}
% Pour la c\'esure des mots
\usepackage[T1]{fontenc}
% Pour traduire par exemple Chapter -> Chapitre
\usepackage{babel}
\usepackage{hegexo}

\newcommand{\ki}{\mathcal{\chi}}

\begin{document}
\cours{Statistique inf�rentielle}
\title{S�rie 1}
\author{Dr S. Varone}
\date{}

\maketitle

%%%%%%%%%%%%%%%%%%%%%%%%%%%%%%%%%%%%%%%%%%%%%%%%%%%%%%%
%%                                                   %%
%%     COURS 2: Loi du Chi-deux, de Student          %%
%%                                                   %%
%%%%%%%%%%%%%%%%%%%%%%%%%%%%%%%%%%%%%%%%%%%%%%%%%%%%%%%


\section{Loi normale}


Soit $Y$, une variable al�atoire suivant une loi normale $\norm(2,9)$. D�terminer \begin{enumerate}
    \item $P(Y < 4.25)$
    \item Pour quelle valeur de $a$ a-t-on $P(Y < a) = 0.64$ ?
    \item $P(Y \in [-1,5])$
\end{enumerate}



\section{Loi du Chi-deux} % Loi du Chi-deux
Soit la variable $Q_{10}$ suivant une loi du chi-2 � 10 degr�s de libert�. D�terminer
\begin{enumerate}
	\item $P(Q_{10} < 15.98)$
	\item $P(Q_{10} > 18.31)$
	\item $P(Q_{10} <18.31)$
	\item $P(15.98 < Q_{10} < 18.31)$
  \item la valeur $a$ telle que $P(Q_{10} < a) = 0.975$
\end{enumerate}



\section{Loi de Student} % Loi de Student
Soit la variable $T_{8}$ suivant une loi de Student � 8 degr�s de libert�. D�terminer
\begin{enumerate}
	\item $P(T_{8} < 0.546)$
	\item $P(T_{8} < - 0.546)$
	\item $ P( - 0.546 < T_{8} < 1.86)$
	\item La valeur $b$ telle que $P(T_{8}> b) = 0.6$
\end{enumerate}

\clearpage

\section{Temps de trajet} % Loi normale
% source: \cite{Groebner2005} p. 202
Une entreprise de transport a �valu� le temps de trajet des citadins pour se rendre � leur travail. Elle en conclut que ce temps de trajet est une variable al�atoire suivant une loi normale, que la moyenne des temps est de 15 minutes et que l'�cart type des temps est de 3.5 minutes. Un citadin affirme que son temps de trajet est de 22 minutes. Le but est de trouver la probabilit� qu'un citoyen ait un temps de trajet de 22 minutes ou plus.
\begin{enumerate}
	\item D�terminer la moyenne et l'�cart-type de la loi suivie par la variable al�atoire X=''Temps de trajet''.
	\item D�finir l'�v�nement d'int�r�t.
	\item Convertir la variable al�atoire en une variable standardis�e $Z$
	\item Trouver la probabilit� associ�e
\end{enumerate}



\section{Arriv�e des vols} % Loi du Chi-deux
	% cf \cite{groebner2008} ex 10.9 p. 392
Un challenge pour les compagnies a�rienne est de respecter les horaires des vols. Une mesure utilis�e est le nombre de minutes d'un vol s�parant l'heure d'arriv�e r�elle de celle annonc�e. La compagnie exige que ses avions arrivent en moyenne � l'heure, avec $\pm 5$ minutes le d�calage entre l'heure d'arriv�e r�elle et celle annonc�e. Ce nombre de minutes est donn�e ci-dessous pour 11 vols:\\
-2 9 10 -3 1 7 -3 5 8 12 4\\
\begin{enumerate}
	\item\label{s2p3i} Calculer la valeur de la statistique $\ki^2=\frac{(n-1)s^2}{\sigma^2}$
	% R
	% x <- c(-2, 9, 10, -3, 1, 7, -3, 5, 8, 12, 4)
	% chideux <- 10*sd(x)^2/25
	% 11.70182
	\item Calculer la probabilit� d'une valeur sup�rieure ou �gale � celle trouv�e en \ref{s2p3i}) (un logiciel est ici n�cessaire pour une bonne approximation)
	% pchisq(chideux,10)
	% $P(\ki^2 \geq 11.70182) =  0.6944917$
\end{enumerate}

\end{document} 