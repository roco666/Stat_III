\documentclass[french,a4paper,12pt]{article}
\usepackage[dvips]{graphicx}
% Pour la c\'esure des mots
\usepackage[T1]{fontenc}
% Pour traduire par exemple Chapter -> Chapitre
\usepackage{babel}
\usepackage{hegexo}

\newcommand{\ki}{\mathcal{\chi}}

\begin{document}
\cours{Statistique inf�rentielle}
\title{Corrig� 1}
\author{Dr S. Varone}
\date{}

\maketitle

%%%%%%%%%%%%%%%%%%%%%%%%%%%%%%%%%%%%%%%%%%%%%%%%%%%%%%%
%%                                                   %%
%%     CORRIGE 2: Loi du Chi-deux, de Student        %%
%%                                                   %%
%%%%%%%%%%%%%%%%%%%%%%%%%%%%%%%%%%%%%%%%%%%%%%%%%%%%%%%


\section{Loi normale} % Loi normale

Soit $Y$ , une variable al�atoire suivant une loi normale $\norm (2, 9)$
	\begin{enumerate}
	  \item D�terminer $P(Y < 4.25)$
	          \begin{eqnarray*}
				     P(Y < 4.25) &=& P(Z<\frac{4.25-2}{3})\\
				     &=& P(Z<0.75)=0.7734
				     \end{eqnarray*}
\begin{verbatim}
pnorm(4.25,2,3)
\end{verbatim}

     \item Pour quelle valeur de $a$ a-t-on $P(Y < a) = 0.64$ ?
		      \begin{eqnarray*}
						P(Y<a)&=&0.64\\
						P(Z<\frac{a-2}{3})&=&0.64\\
						\frac{a-2}{3}&\approx& 0.36\\
						a&=&3\cdot 0.36+2=3.08
					\end{eqnarray*}
\begin{verbatim}
qnorm(0.64,2,3)
\end{verbatim}
	      \item $P(Y \in [-1,5])$

					  $\begin{array}{rcl}
					   P(Y<5)-P(Y<-1) &=& P(Z<\frac{5-2}{3})-P(Z<\frac{-1-2}{3}) \\
                        &=& P(Z<1)-P(Z<-1) \\
                        &=& 2P(Z<1)-1 = 2\cdot 0.8413 -1 = 0.6826
					  \end{array}$
\begin{verbatim}
pnorm(5,2,3)-pnorm(-1,2,3)
\end{verbatim}
  \end{enumerate}




\section{Loi du Chi-deux} % Loi du Chi-deux
Soit la variable $Q_{10}$ suivant une loi du chi-2 � 10 degr�s de libert�. D�terminer
\begin{enumerate}
	\item $P(Q_{10} < 15.98)=0.9$
\begin{verbatim}
pchisq(15.98,10)
\end{verbatim}
	\item $P(Q_{10} > 18.31)=0.05$
\begin{verbatim}
1-pchisq(18.31,10)
\end{verbatim}
	\item $P(Q_{10} <18.31)=0.95$
	\item $P(15.98 < Q_{10} < 18.31)=P( Q_{10} < 18.31)-P(Q_{10}<15.98)=0.95-0.9=0.05$
  \item la valeur $a$ telle que $P(Q_{10} < a) = 0.975$, $a=20.48$
\begin{verbatim}
qchisq(0.975,10)
\end{verbatim}
\end{enumerate}



\section{Loi de Student} % Loi de Student
Soit la variable $T_{8}$ suivant une loi de Student � 8 degr�s de libert�. D�terminer
\begin{enumerate}
	\item $P(T_{8} < 0.546)= 1-P(T_{8} > 0.546) = 0.7$
\begin{verbatim}
pt(0.546,8)
\end{verbatim}
	\item $P(T_{8} < - 0.546) = P(T_{8} > 0.546) =0.3$
	\item $ P( - 0.546 < T_{8} < 1.86)$
		\begin{eqnarray*}
			P( - 0.546 < T_{8} < 1.86)&=&P( T_{8} < 1.86)-P( T_{8} <  - 0.546)\\
			&=&P( T_{8} < 1.86)- (1-P( T_{8} <  0.546))\\
			&=&0.95-(1-0.7)=0.65
		\end{eqnarray*}
\begin{verbatim}
pt(1.86,8)-pt(-0.546,8)
\end{verbatim}
	\item La valeur $b$ telle que $P(T_{8}> b) = 0.6$
		 \begin{eqnarray*}
			P(T_{8}> b) &=& 0.6\\
			P(T_{8}<- b) &=& 0.6\\
			-b&=&0.2619\\
			b&=&-0.26219
			\end{eqnarray*}
\begin{verbatim}
-qt(0.6,8)
\end{verbatim}
\end{enumerate}




\section{Temps de trajet} % Loi normale
% source: \cite{Groebner2005} p. 202
Une entreprise de transport a �valu� le temps de trajet des citadins pour se rendre � leur travail. Elle en conclut que ce temps de trajet est une variable al�atoire suivant une loi normale, que la moyenne des temps est de 15 minutes et que l'�cart type des temps est de 3.5 minutes. Un citadin affirme que son temps de trajet est de 22 minutes. Le but est de trouver la probabilit� qu'un citoyen ait un temps de trajet de 22 minutes ou plus.
\begin{enumerate}
	\item D�terminer la moyenne et l'�cart-type de la loi suivie par la variable al�atoire X=''Temps de trajet''.
	\\ $\mu = 15$ et $\sigma = 3.5$
	\item D�finir l'�v�nement d'int�r�t.
	\\ $P(X\geq 22)$
	\item Convertir la variable al�atoire en une variable standardis�e $Z$
	\\ $Z=\frac{22-15}{3.5}=2.0$
	\item Trouver la probabilit� associ�e
	\\ $P(X\geq 22)=P(Z\geq 2) = 1-0.9772 = 0.0228$
\end{enumerate}
\begin{verbatim}
1-pnorm(22,15,3.5)
\end{verbatim}


\section{Arriv�e des vols} % Loi du Chi-deux
	% cf \cite{groebner2008} ex 10.9 p. 392
Un challenge pour les compagnies a�rienne est de respecter les horaires des vols. Une mesure utilis�e est le nombre de minutes d'un vol s�parant l'heure d'arriv�e r�elle de celle annonc�e. La compagnie exige que ses avions arrivent en moyenne � l'heure, avec $\pm 5$ minutes le d�calage entre l'heure d'arriv�e r�elle et celle annonc�e. Ce nombre de minutes est donn�e ci-dessous pour 11 vols:\\
-2 9 10 -3 1 7 -3 5 8 12 4\\
\begin{enumerate}
	\item\label{s2p3i} Calculer la valeur de la statistique $\ki^2=\frac{(n-1)s^2}{\sigma^2}$
\\Avec R
\begin{verbatim}
x <- c(-2, 9, 10, -3, 1, 7, -3, 5, 8, 12, 4)
chideux <- 10*sd(x)^2/25
\end{verbatim}
$\ki^2= = 11.70182$
	\item Calculer la probabilit� d'une valeur sup�rieure ou �gale � celle trouv�e en \ref{s2p3i}) (un logiciel est ici n�cessaire pour une bonne approximation)
\begin{verbatim}
x <- c(-2, 9, 10, -3, 1, 7, -3, 5, 8, 12, 4)
chideux <- 10*sd(x)^2/25
pchisq(chideux,10)
\end{verbatim}
$P(\ki^2 \geq 11.70182) =  0.6944917$
\end{enumerate}
\end{document} 