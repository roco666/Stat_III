\documentclass[french,a4paper,12pt]{article}
\usepackage[dvips]{graphicx}
% Pour la c\'esure des mots
\usepackage[T1]{fontenc}
% Pour traduire par exemple Chapter -> Chapitre
\usepackage{babel}
\usepackage{hegexo}


\begin{document}
\cours{Statistique inf�rentielle}
\title{S�rie 10}
\author{Dr S. Varone}
\date{}

\maketitle

%%%%%%%%%%%%%%%%%%%%%%%%%%%%%%%%%%%%%%%%%%%%%%%%%%%%%%%
%%                                                   %% 
%%     COURS 10: Test de la m�diane                  %%
%%                                                   %% 
%%%%%%%%%%%%%%%%%%%%%%%%%%%%%%%%%%%%%%%%%%%%%%%%%%%%%%%

\section{Lampe de beamer} % Test de la m�diane
	% \cite{groebner2006} ex 16.5 p. 674
Un fabricant de beamer affirme que les lampes de ses produits ont une dur�e de vie m�diane de 4000 heures. Un magazine de consommateurs d�cide de v�rifier les dires de ce fabricant et teste la dur�e de vie de 12 beamers.

\begin{tabular}{*{6}{c}}
1973 & 4838 & 3805 & 4494 & 4738 & 5249\\
4459 & 4098 & 4722 & 5894 & 3322 & 4800
\end{tabular}

En utilisant un niveau de signification de 0.05, quelle conclusion le magazine de consommateurs peut-il avoir ?


{\it Indication}: Voici le r�sultat donn� par le logiciel R en appliquant le test des rangs sign�s de Wilcoxon:

\begin{verbatim}
        Wilcoxon signed rank test

data:  x 
V = 59, p-value = 0.1294
alternative hypothesis: true location is not equal to 4000 
95 percent confidence interval:
 3710.0 4985.5 
sample estimates:
(pseudo)median 
       4485.25 
\end{verbatim}



%	\begin{enumerate}
%		\item Le param�tre est $\tilde{\mu}=4000$
%		\item $$\begin{array}{ll}
%	  H_0 &: p \tilde{\mu}\leq4000 \\
%	  H_1 :& \tilde{\mu}>4000
%	 \end{array}$$
%	  \item Le niveau de signification est $\alpha=0.05$
%	  \item La r�gion critique est $[61 ; \infty [$
%	  \item Calcul de la statistique
%	  
%\begin{tabular}{*{6}{c}}
%Donn�es & Diff�rences & Abs(Diff�rences) & Rang	& Rang + & 	Rang -\\
%1973 & -2027 & 2027 & 12 &  & 12\\
%4838 & 838 & 838 & 9 & 9 &\\
%3805 & -195 & 195 & 2 & &  2\\
%4494 & 494 & 494 & 4 & 4 & \\
%4738 & 738 & 738 & 7 & 7 & \\
%5249 & 1249 & 1249 & 10 & 10 & \\
%4459 & 459 & 459 & 3 & 3 & \\
%4098 & 98 & 98 & 1 & 1 & \\
%4722 & 722 & 722 & 6 & 6 & \\
%5894 & 1894 & 1894 & 11 & 11 & \\
%3322 & -678 & 678 & 5 &  & 5\\
%4800 & 800 & 800 & 8 & 8 & \\
% & & & & W=	59 &	
%\end{tabular}
%	  
%	  \item Comme $W=59<61=W_0$ l'hypoth�se $H_0$ n'est pas rejet�e
%	  \item La conclusion est qu'il n'y a pas suffisamment d'�vidence pour rejeter l'affirmation du fabricant
%	\end{enumerate}


\end{document}