%\documentclass[style=heg, mode=print]{powerdot}
\documentclass[style=heg]{powerdot}

\usepackage{amsfonts}
\usepackage{pst-plot}
	% Pour la c\'esure des mots
\usepackage[T1]{fontenc}
	% Pour traduire par exemple Chapter -> Chapitre
\usepackage[frenchb]{babel}
\frenchbsetup{StandardLayout}
	% Chemin des graphiques
\graphicspath{{../fig/}}
\usepackage{hyperref}
% Definitions hyperref
\hypersetup{colorlinks,%
            citecolor=black,%
            filecolor=black,%
            linkcolor=black,%
            urlcolor=blue}

% Definitions Maths
\newcommand{\IN} {\mathbb{N}}
\newcommand{\ZZ} {\mathbb{Z}}
\newcommand{\IR} {\mathbb{R}}
\newcommand{\IB} {\mathbb{B}}
\def\X{\mathop{\lower 2pt\hbox{\large{\textsf X}}}}
\newcommand{\var} {{\rm var}}
\newcommand{\cov} {{\rm cov}}
\newcommand{\corr} {{\rm corr}}
\newcommand{\E}{{\rm E}}
\newcommand{\biais}{{\rm biais}}
\newcommand{\med}{{\rm med}}
\newcommand{\prob}{{\rm Pr}}
\newcommand{\bin}{\mathcal{B}}
\newcommand{\binneg}{\mathcal{BN}}
\newcommand{\pois}{\mathcal{P}}
\newcommand{\hyperg}{\mathcal{H}}
\newcommand{\norm}{\mathcal{N}}
\newcommand{\unif}{\mathcal{U}}
\newcommand{\st}{\mathcal{T}}
\newcommand{\ki}{\mathcal{\chi}}

	%\Logo{\includegraphics[scale=0.35]{heg}}
\title{Test de la moyenne}
\author{Dr. Sacha Varone}
\date{}
\pdsetup{
rf={\tiny stat III - compl�ment 8}
}

\begin{document}
%\maketitle

\begin{slide}[toc=VaR]{Value at Risk}
La \emph{Value at Risk} (VaR) est d�finie comme la perte maximale sur un horizon donn� $T$, avec un niveau de confiance $1-\alpha$.\\[5mm]
Consid�rons une variable al�atoire $X\sim \norm(\mu,\sigma^2)$
$$P\left( X\leq \mbox{VaR}(\alpha,T)\right) = \alpha \quad\Leftrightarrow\quad P\left( X\geq \mbox{VaR}(\alpha,T)\right) = 1-\alpha$$
\end{slide}

\begin{slide}[toc=]{VaR en Finance}
La notion de risque financier peut �tre estim� � l'aide de l'�cart type annuelle des performances financi�res, appel� dans le jargon bancaire la \emph{volatilit�}. L'horizon est quant � lui souvent donn� en jours (1, 10, ...), qui est � mettre en regard du nombre de jours ouvrables consid�r� dans la branche (250 ou 252 g�n�ralement). Consid�rons $X$ une variable al�atoire indiquant les rendements, suivant une m�me loi normale $\norm(\mu,\sigma^2)$ sur une seule p�riode.  Alors, sur la p�riode $T$, les rendements sont �galement gausssiens de moyenne $\mu T$ et de variance $\sigma^2 T$. La variable centr�e r�duite s'�crit :
$$z_{\alpha}=\frac{\mbox{VaR}(\alpha,T)-\mu T}{\sigma\sqrt{T}}$$
Et donc 
$$\mbox{VaR}(\alpha,T) = \mu T + z_{\alpha}\sigma\sqrt{T}$$
\end{slide}

\begin{slide}[toc=]{Exemple}
{\it\small Source: \href{http://www.ubs.com/1/e/investors/annualreporting.html}{Rapport annuel 2007 de l'UBS}\\
version anglaise, ''Risk, Treasury and Capital Management'', p. 39 }\\
La position du secteur Banque d'investissement de l'UBS, sur un horizon de 1 jour, � un niveau de confiance de 99\%, en utilisant des donn�es sur 5 ans, est de 
\begin{itemize}
	\item 160 millions en 2007
	\item 169 millions en 2006
\end{itemize} 
\end{slide}

\end{document}

%%% Local Variables:
%%% mode: latex
%%% TeX-master: t
%%% End:
