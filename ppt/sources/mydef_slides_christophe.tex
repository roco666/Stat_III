
% Definitions Style
\newenvironment{defi}[1][]%
{\begin{flushleft}\textbf{D�finition }\\}%
{\end{flushleft}\vspace{1mm}}
%
\newenvironment{pro}[1][]%
{\begin{flushleft}\textbf{Propri�t� }}%
{\end{flushleft}\vspace{1mm}}
%
\newenvironment{cor}[1][]%
{\begin{flushleft}\textbf{Corollaire }}%
{\end{flushleft}\vspace{1mm}}
%
\newenvironment{lem}[1][]%
{\begin{flushleft}\textbf{Lemme }}%
{\end{flushleft}\vspace{1mm}}
%
\newenvironment{theo}[1][]%
{\begin{flushleft}\textbf{Th�or�me }}%
{\end{flushleft}\vspace{1mm}}
%
\newenvironment{proof}
{\begin{quote}\textbf{Preuve :}\\} {\\$\sharp$\end{quote}}
%
\newenvironment{ex}
{\textit{Exemple :}\begin{quote}} {\end{quote}}
%
\newenvironment{rem}
{\textit{Remarque : }} {}
%
\newenvironment{exo}
{\textit{Exercice :}\begin{quote}} {\end{quote}}
%
\newenvironment{algoC}
{\footnotesize\setlength{\parskip}{0.2em}\begin{multicols}{2}}
{\end{multicols}}



\let\ds=\displaystyle
\let\ts=\textstyle
\let\emph=\textit


%couleurs

\newcommand{\Rouge}[1]{{\red #1}}
\renewcommand{\Vert}[1]{{\green #1}}
\newcommand{\Bleu}[1]{{\blue #1}}


% Definitions Maths
\newcommand{\C}{{\mathbb C}}
\newcommand{\R}{{\mathbb R}}
\newcommand{\Q}{{\mathbb Q}}
\newcommand{\Z}{{\mathbb Z}}
\newcommand{\N}{{\mathbb N}}
\newcommand{\IN} {\mathbb{N}}
\newcommand{\ZZ} {\mathbb{Z}}
\newcommand{\IR} {\mathbb{R}}
\newcommand{\IB} {\mathbb{B}}

\def\X{\mathop{\lower 2pt\hbox{\large{\textsf X}}}}

\DeclareMathOperator{\E}{E}
\DeclareMathOperator{\med}{med}
\DeclareMathOperator{\prob}{P}

\newcommand{\bin}{\mathcal{B}}
\newcommand{\binneg}{\mathcal{BN}}
\newcommand{\pois}{\mathcal{P}}
\newcommand{\hyperg}{\mathcal{H}}
\newcommand{\norm}{\mathcal{N}}
%\newcommand{\expo}{\mathcal{Exp}}
\newcommand{\unif}{\mathcal{U}}
\newcommand{\st}{\mathcal{T}}
\newcommand{\ki}{\mathcal{\chi}}


\DeclareMathOperator{\Var}{Var}
\DeclareMathOperator{\Cov}{Cov}
\DeclareMathOperator{\Corr}{Corr}
\DeclareMathOperator{\Biais}{Biais}
\DeclareMathOperator{\EQM}{EQM}
\DeclareMathOperator{\sas}{sas}
\DeclareMathOperator{\boot}{bootstrap}
\DeclareMathOperator{\jack}{jackknife}
\DeclareMathOperator{\str}{str} % \st = conflit avec Student
\DeclareMathOperator{\SC}{s.c.}
\DeclareMathOperator{\prop}{prop}
\DeclareMathOperator{\stprop}{stprop}
\DeclareMathOperator{\opt}{opt}
\DeclareMathOperator{\inter}{inter}
\DeclareMathOperator{\intra}{intra}
\DeclareMathOperator{\post}{post}
\DeclareMathOperator{\quot}{quot}
\DeclareMathOperator{\reg}{reg}

\let\Ra=\Rightarrow
\let\Lra=\Longrightarrow
\let\Llra=\Longleftrightarrow


\newcommand{\cad}{c'est-�-dire\xspace}
\newcommand{\Cad}{C'est-�-dire\xspace}
\newcommand{\ie}{\textit{i.e.}\xspace}

\newcommand{\place}[3]{\vspace*{#1cm}\hspace*{#2cm}#3} 